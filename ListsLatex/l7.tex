\documentclass[18pt]{extarticle}

\usepackage{amsmath}
\usepackage{amssymb}
\usepackage{enumitem}
\usepackage[polish]{babel}
\usepackage[T1]{fontenc}
\usepackage[utf8]{inputenc}
\usepackage[margin=1.9cm]{geometry}
\usepackage[edges]{forest}
\usepackage{listings}
\usepackage{multicol}
\usepackage{tgbonum}
\usepackage{xcolor}

\begin{document}
\large
\fontfamily{cmss}\selectfont

\title{Algorytmy i struktury danych (Lista 6)}
\date{}
\maketitle

\paragraph{Zadanie 1} Jakie informacje przechowujemy w węźle B-drzewa? Podaj definicję B-drzewa. \\

\begin{lstlisting}
    struct BTnode:
        int t;              // minimum degree
        int n;              // current number of keys
        int* keys;          // array of keys in non-decreasing order
        bool leaf;          // is it a leaf?
        BTnode** children;
    end
\end{lstlisting}
Inne założenia (oprócz zawartych w komentarzach):
\begin{enumerate}
    \item Węzeł wewnętrzny zawiera $n+1$ wskaźników do synów.
    \item Klucze rozdzielają dzieci na przedziały (n+1).
    \item Każdy węzeł różny od korzenia musi mieć co najmniej $t-1$ kluczy i co najwyżej $2t-1$ kluczy. (korzeń może mieć od $1$ do $2t-1$ kluczy)
    \item Wszystkie liście leżą na tej samej wysokości równym $h$.
\end{enumerate}

\paragraph{Zadanie 2} (2 pkt.) Udowodnij, że żadna z poniższych operacji wykonana na drzewie spełniającym wszystkie warunki B-drzewa, nie prowadzi do ich naruszenia.
\begin{enumerate}[label=(\alph*)]
    \item \verb|split_child|, przesuwająca środkowy klucz (medianę) z węzła o $2t-1$ kluczach do rodzica, który ma mniej niż $2t-1$ kluczy, a klucze i dzieci na prawo od mediany -- do nowego brata dodanego po prawej stronie dzielonego węzła.
        Skoro mogliśmy przejść do dziecka to znaczy, że rodzic był niepełen i może przyjąć przynajmniej jeden klucz. Więc nie narusza się maksymalnej liczby kluczy w węźle. \\
        Węzeł dzielony miał $2t-1$ kluczy, więc bez tego odjętego zostaja nam dwa zestawy po minimalnej liczbie kluczy.
    \item \verb|unsplit_child| odwrotna do \verb|split_child|, sklejająca dwa sąsiednie węzły o minimalnej liczbie
          kluczy $t-1$ oraz klucz stojący w rodzicu między nimi w jeden nowy węzeł. Zakładamy, że rodzic ma co
          najmniej $t$ kluczy lub jest korzeniem.
    \item \verb|borrow_from_sibling|, rotacja przenosząca do węzła o minimalnej $t-1$ liczbie kluczy, który ma
          prawego brata z co najmniej $t$ kluczami, klucz stojący w rodzicu między braćmi i wpisująca na jego miejsce jego miejsce pierwszy klucz brata. Jakie operacje na dzieciach należy dodatkowo wykonać?
\end{enumerate}

\paragraph{Zadanie 3} W B-drzewie o $t=10$ podaj wzory i wyniki numeryczne określające:
\begin{enumerate}[label=(\alph*)]
    \item ile kluczy może zawierać korzeń (podaj przedział), \\
    Korzeń zawiera od 1 do 19 kluczy. (max $2t-1$)
    \item ile dzieci może mieć korzeń (podaj przedział), \\
    Korzeń może mieć od 2 do 20 dzieci. (min $t$ max $2t$)
    \item ile kluczy może mieć potomek korzenia (podaj przedział), \\
    Potomek korzenia może mieć od 9 do 19 kluczy. (min $t-1$ max $2t-1$)
    \item ile dzieci może mieć potomek korzenia (podaj przedział), \\
    Potomek korzenia może mieć od 10 do 20 dzieci. (min $t$ max $2t$)
    \item ile maksymalnie węzłów może być na $k$-tym poziomie (przyjmując, że korzeń to poziom $0$) \\
    Na k-tym poziomie może być maksymalnie $(2t)^k$ węzłów.
    \item ile łącznie kluczy może być na $k$-tym poziomie (podaj przedział). \\
    Nie licząc korzenia dla którego minimum to 1 klicz to na k-tym poziomie może być od $2(t-1)t^{k-1}$ do $(2t-1)(2t)^{k}$ kluczy. (min $(2min)t^{k-1}$ max $(max)t^{k}$
\end{enumerate}

\paragraph{Zadanie 4} Jaka jest minimalna, a jaka maksymalna liczba kluczy w B-drzewie mającym $h$ poziomów, przy ustalonej wartości parametru $t$ (patrz Cormen). \\

Minimalna: \\
Gdy korzeń zawiera 1 klucz, a na każdym z pozostałych poziomiomów każdy węzeł zawiera $t-1$ kluczy (minimum). \\
Na poziomie 1 mamy 2 węzły -> $2(t-1)$ kluczy. \\
Z czego wynika, że na kolejnym poziomie mamy 2t węzłów, na następnym $2t^2$ itd. \\
W takim układzie na $h$ poziomie mamy $2t^{h-1}$ węzłów, a każdy z nich zawiera $t-1$ kluczy. \\
Stąd: \\
\begin{align*}
    n &\geq 1 + 2(t-1)(2t + 2t^2 + ... + 2t^{h-1}) \\
    &= 1 + (t-1)\sum_{i=1}^{h}2t^{i-1} \\
    &= 1 + 2(t-1)\frac{t^h-1}{t-1} \\
    &= 2t^h - 1
\end{align*}

Maksymalna: \\
Gdy korzeń ma maksymalną ilość kluczy $2t-1$, a na każdym z pozostałych poziomów każdy węzeł zawiera $2t-1$ kluczy (maksimum). \\
Na poziomie 1 mamy max $2t-1$ kluczy razy wezły 2t -> $(2t-1)2t$. \\
Z czego wynika, że na kolejnym poziomie mamy $(2t-1)2t^2$ kluczy, na następnym $(2t-1)2t^3$ itd. \\
W takim układzie na $h$ poziomie mamy $(2t-1)2t^{h}$ kluczy, a każdy z nich zawiera $2t-1$ kluczy. \\
Stąd: \\
\begin{align*}
    n &\leq (2t-1) + (2t-1)(2t + {(2t)}^2 + ... + {(2t)}^{h}) \\
    &= (2t-1) + (2t-1)\sum_{i=1}^{h}{(2t)}^{i} \\
    &= (2t-1)\sum_{i=0}^{h}{(2t)}^{i} \\
    &= (2t-1)\frac{{(2t)}^{h+1}-1}{2t-1} \\
    &= {(2t)}^{h+1}-1
\end{align*}

\paragraph{Zadanie 5} Podano na rysunku B-drzewo o $t=2$:
\begin{center}
    \begin{tikzpicture}
        \tikzstyle{bplus}=[rectangle split, rectangle split horizontal, rectangle split ignore empty parts, draw]
        \tikzstyle{every node}=[bplus]
        \tikzstyle{level 1}=[sibling distance=60mm]
        \tikzstyle{level 2}=[sibling distance=15mm]
        \node {9} [->]
        child {node {7}
                child {node {6}}
                child {node {8}}
            }
        child {node {11 \nodepart{two} 14 \nodepart{three} 19}
                child {node {10}}
                child {node {12 \nodepart{two} 13}}
                child[sibling distance=25mm] {node {15 \nodepart{two} 16 \nodepart{three} 17}}
                child[sibling distance=20mm] {node {20}}
            };
    \end{tikzpicture}
    \begin{enumerate}[label=-]
        \item usuń z tego drzewa 7.
        \item do drzewa widocznego powyżej dodaj 18.
    \end{enumerate}
    Próbujemy zastąpić 7 przez klucz z dziecka, ale brakuje nam liścia, więc łączymy dzieci (6 i 8) w jeden węzeł. \\
    \begin{tikzpicture}
        \tikzstyle{bplus}=[rectangle split, rectangle split horizontal, rectangle split ignore empty parts, draw]
        \tikzstyle{every node}=[bplus]
        \tikzstyle{level 1}=[sibling distance=60mm]
        \tikzstyle{level 2}=[sibling distance=15mm]
        \node {9} [->]
        child {node {-}
                child {node {6 \nodepart{two} 8}}
            }
        child {node {11 \nodepart{two} 14 \nodepart{three} 19}
                child {node {10}}
                child {node {12 \nodepart{two} 13}}
                child[sibling distance=25mm] {node {15 \nodepart{two} 16 \nodepart{three} 17}}
                child[sibling distance=20mm] {node {20}}
            };
    \end{tikzpicture} \\
    Nowym rodzicem będzie 9 a korzeniem 11 pożyczona z brata 7. Najmłodsze dziecko 11 przerzucamy jako najstarsze dziecko 9. \\
    \begin{tikzpicture}
        \tikzstyle{bplus}=[rectangle split, rectangle split horizontal, rectangle split ignore empty parts, draw]
        \tikzstyle{every node}=[bplus]
        \tikzstyle{level 1}=[sibling distance=60mm]
        \tikzstyle{level 2}=[sibling distance=15mm]
        \node {11} [->]
        child {node {9}
            child {node {6 \nodepart{two} 8}}
            child {node {10}}
        }
        child {node {14 \nodepart{three} 19}
            child[sibling distance=25mm] {node {12 \nodepart{two} 13}}
            child {node {15 \nodepart{two} 16 \nodepart{three} 17}}
            child[sibling distance=25mm] {node {20}}
        };
    \end{tikzpicture} \\
    Dodanie 18: \\
    \begin{tikzpicture}
        \tikzstyle{bplus}=[rectangle split, rectangle split horizontal, rectangle split ignore empty parts, draw]
        \tikzstyle{every node}=[bplus]
        \tikzstyle{level 1}=[sibling distance=60mm]
        \tikzstyle{level 2}=[sibling distance=15mm]
        \node {9} [->]
        child {node {7}
            child {node {6}}
            child {node {8}}
        }
        child {node {11 \nodepart{two} 14 \nodepart{three} 19}
            child {node {10}}
            child {node {12 \nodepart{two} 13}}
            child[sibling distance=25mm] {node {15 \nodepart{two} 16 \nodepart{three} 17}}
            child[sibling distance=20mm] {node {20}}
        };
    \end{tikzpicture} \\

    \begin{tikzpicture}
        \tikzstyle{bplus}=[rectangle split, rectangle split horizontal, rectangle split ignore empty parts, draw]
        \tikzstyle{every node}=[bplus]
        \tikzstyle{level 1}=[sibling distance=60mm]
        \tikzstyle{level 2}=[sibling distance=15mm]
        \node {9 \nodepart{two} 14} [->]
        child {node {7}
            child {node {6}}
            child {node {8}}
        }
        child {node {11 \nodepart{two} 18}
            child {node {10}}
            child {node {12 \nodepart{two} 13}}
        }
        child {node {19}
            child[sibling distance=25mm] {node {15 \nodepart{two} 16 \nodepart{three} 17}}
            child[sibling distance=20mm] {node {20}}
        };
    \end{tikzpicture} \\

    \begin{tikzpicture}
        \tikzstyle{bplus}=[rectangle split, rectangle split horizontal, rectangle split ignore empty parts, draw]
        \tikzstyle{every node}=[bplus]
        \tikzstyle{level 1}=[sibling distance=60mm]
        \tikzstyle{level 2}=[sibling distance=15mm]
        \node {9 \nodepart{two} 14} [->]
        child {node {7}
            child {node {6}}
            child {node {8}}
        }
        child {node {11}
            child {node {10}}
            child {node {12 \nodepart{two} 13}}
        }
        child {node {19}
            child[sibling distance=25mm] {node {15 \nodepart{two} 16 \nodepart{three} 17 \nodepart{four} 18}}
            child[sibling distance=20mm] {node {20}}
        };
    \end{tikzpicture} \\

    \begin{tikzpicture}
        \tikzstyle{bplus}=[rectangle split, rectangle split horizontal, rectangle split ignore empty parts, draw]
        \tikzstyle{every node}=[bplus]
        \tikzstyle{level 1}=[sibling distance=60mm]
        \tikzstyle{level 2}=[sibling distance=15mm]
        \node {9 \nodepart{two} 14} [->]
        child {node {7}
            child {node {6}}
            child {node {8}}
        }
        child {node {11}
            child {node {10}}
            child {node {12 \nodepart{two} 13}}
        }
        child {node {17 \nodepart{two} 19}
            child[sibling distance=20mm] {node {15 \nodepart{two} 16}}
            child{node {18}}
            child[sibling distance=20mm] {node {20}}
        };
    \end{tikzpicture}
\end{center}

\paragraph{Zadanie 6} (2 pkt.) Do pustego B-drzewa o $t=2$ wstaw kolejno $22$ litery swojego imienia i nazwiska oraz adresu. Następnie usuń w tej samej kolejności w jakiej były wstawiane.

\pagebreak
\paragraph{Zadanie 7} Narysuj B-drzewo o $t=3$ zawierające dokładnie $17$ kluczy na trzech poziomach: korzeń, jego dzieci i wnuki. Następnie usuń z tego drzewa korzeń.

\begin{center}
    \begin{tikzpicture}
        \tikzstyle{bplus}=[rectangle split, rectangle split horizontal, rectangle split ignore empty parts, draw]
        \tikzstyle{every node}=[bplus]
        \tikzstyle{level 1}=[sibling distance=60mm]
        \tikzstyle{level 2}=[sibling distance=15mm]
        \node {9} [->]
        child {node {3 \nodepart{two} 6}
            child {node {1 \nodepart{two} 2}}
            child {node {4 \nodepart{two} 5}}
            child {node {7 \nodepart{two} 8}}
        }
        child {node {12 \nodepart{two} 15}
            child {node {10 \nodepart{two} 11}}
            child {node {13 \nodepart{two} 14}}
            child {node {16 \nodepart{two} 17}}
        };
    \end{tikzpicture}
    \begin{tikzpicture}
        \tikzstyle{bplus}=[rectangle split, rectangle split horizontal, rectangle split ignore empty parts, draw]
        \tikzstyle{every node}=[bplus]
        \tikzstyle{level 1}=[sibling distance=60mm]
        \tikzstyle{level 2}=[sibling distance=15mm]
        \node {10} [->]
        child {node {3 \nodepart{two} 6}
            child {node {1 \nodepart{two} 2}}
            child {node {4 \nodepart{two} 5}}
            child {node {7 \nodepart{two} 8}}
        }
        child {node {12 \nodepart{two} 15}
            child {node {11}}
            child {node {13 \nodepart{two} 14}}
            child {node {16 \nodepart{two} 17}}
        };
    \end{tikzpicture}\\
    11 nie spełnia sama warunku B-drzewa, więc musimy pożyczyć od rodzica 12, jednak wtedy z 15 dzieje się to samo. Nie mamy jak pożyczyć od rodzica ani od brata. Dlatego musimy złączy je w jeden węzeł. \\ 
    \begin{tikzpicture}
        \tikzstyle{bplus}=[rectangle split, rectangle split horizontal, rectangle split ignore empty parts, draw]
        \tikzstyle{every node}=[bplus]
        \tikzstyle{level 1}=[sibling distance=30mm]
        \tikzstyle{level 2}=[sibling distance=15mm]
        \node {3 \nodepart{two} 6 \nodepart{three} 10 \nodepart{four} 15} [->]
        child {node {1 \nodepart{two} 2}}
        child {node {4 \nodepart{two} 5}}
        child {node {7 \nodepart{two} 8}}
        child {node {11 \nodepart{two} 12 \nodepart{three} 13 \nodepart{four} 14}}
        child {node {16 \nodepart{two} 17}};
    \end{tikzpicture}
\end{center}

\end{document}