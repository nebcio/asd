\documentclass[18pt]{extarticle}

\usepackage{amsmath}
\usepackage{amssymb}
\usepackage{enumitem}
\usepackage[polish]{babel}
\usepackage[T1]{fontenc}
\usepackage[utf8]{inputenc}
\usepackage[margin=1.9cm]{geometry}
\usepackage[edges]{forest}
\usepackage{listings}
\usepackage{multicol}
\usepackage{tgbonum}
\usepackage{xcolor}

\begin{document}
\large
\fontfamily{cmss}\selectfont

\title{Algorytmy i struktury danych (Lista 6)}
\date{}
\maketitle

\paragraph{Zadanie 1} Jakie informacje przechowujemy w węźle B-drzewa? Podaj definicję B-drzewa. \\

\begin{lstlisting}
    struct BTnode:
        int t;              // minimum degree
        int n;              // current number of keys
        int* keys;          // array of keys in non-decreasing order
        bool leaf;          // is it a leaf?
        BTnode** children;
    end
\end{lstlisting}
Inne założenia (oprócz zawartych w komentarzach):
\begin{enumerate}
    \item Węzeł wewnętrzny zawiera $n+1$ wskaźników do synów.
    \item Klucze rozdzielają dzieci na przedziały (n+1).
    \item Każdy węzeł różny od korzenia musi mieć co najmniej $t-1$ kluczy i co najwyżej $2t-1$ kluczy. (korzeń może mieć od $1$ do $2t-1$ kluczy)
    \item Wszystkie liście leżą na tej samej wysokości równym $h$.
\end{enumerate}

\paragraph{Zadanie 2} (2 pkt.) Udowodnij, że żadna z poniższych operacji wykonana na drzewie spełniającym wszystkie warunki B-drzewa, nie prowadzi do ich naruszenia.
\begin{enumerate}[label=(\alph*)]
    \item \verb|split_child|, przesuwająca środkowy klucz (medianę) z węzła o $2t-1$ kluczach do rodzica, który ma
          mniej niż $2t-1$ kluczy, a klucze i dzieci na prawo od mediany -- do nowego brata dodanego po prawej stronie dzielonego węzła.
    \item \verb|unsplit_child| odwrotna do \verb|split_child|, sklejająca dwa sąsiednie węzły o minimalnej liczbie
          kluczy $t-1$ oraz klucz stojący w rodzicu między nimi w jeden nowy węzeł. Zakładamy, że rodzic ma co
          najmniej $t$ kluczy lub jest korzeniem.
    \item \verb|borrow_from_sibling|, rotacja przenosząca do węzła o minimalnej $t-1$ liczbie kluczy, który ma
          prawego brata z co najmniej $t$ kluczami, klucz stojący w rodzicu między braćmi i wpisująca na jego miejsce jego miejsce pierwszy klucz brata. Jakie operacje na dzieciach należy dodatkowo wykonać?
\end{enumerate}

\paragraph{Zadanie 3} W B-drzewie o $t=10$ podaj wzory i wyniki numeryczne określające:
\begin{enumerate}[label=(\alph*)]
    \item ile kluczy może zawierać korzeń (podaj przedział),
    \item ile dzieci może mieć korzeń (podaj przedział),
    \item ile kluczy może mieć potomek korzenia (podaj przedział),
    \item ile dzieci może mieć potomek korzenia (podaj przedział),
    \item ile maksymalnie węzłów może być na $k$-tym poziomie (przyjmując, że korzeń to poziom $0$)
    \item ile łącznie kluczy może być na $k$-tym poziomie (podaj przedział).
\end{enumerate}

\paragraph{Zadanie 4} Jaka jest minimalna, a jaka maksymalna liczba kluczy w B-drzewie mającym $h$ poziomów, przy ustalonej wartości parametru $t$ (patrz Cormen).

\paragraph{Zadanie 5} Podano na rysunku B-drzewo o $t=2$:
\begin{center}
    \begin{tikzpicture}
        \tikzstyle{bplus}=[rectangle split, rectangle split horizontal, rectangle split ignore empty parts, draw]
        \tikzstyle{every node}=[bplus]
        \tikzstyle{level 1}=[sibling distance=60mm]
        \tikzstyle{level 2}=[sibling distance=15mm]
        \node {9} [->]
        child {node {7}
                child {node {6}}
                child {node {8}}
            }
        child {node {11 \nodepart{two} 14 \nodepart{three} 19}
                child {node {10}}
                child {node {12 \nodepart{two} 13}}
                child[sibling distance=25mm] {node {15 \nodepart{two} 16 \nodepart{three} 17}}
                child[sibling distance=20mm] {node {20}}
            };
        \end{tikzpicture}
        \begin{enumerate}[label=-]
            \item usuń z tego drzewa 7.
            \item do drzewa widocznego powyżej dodaj 18.
        \end{enumerate}
\end{center}

\paragraph{Zadanie 6} (2 pkt.) Do pustego B-drzewa o $t=2$ wstaw kolejno $22$ litery swojego imienia i nazwiska oraz adresu. Następnie usuń w tej samej kolejności w jakiej były wstawiane.

\paragraph{Zadanie 7} Narysuj B-drzewo o $t=3$ zawierające dokładnie $17$ kluczy na trzech poziomach: korzeń, jego dzieci i wnuki. Następnie usuń z tego drzewa korzeń.


\end{document}