\documentclass[18pt]{extarticle}

\usepackage{amsmath}
\usepackage{amssymb}
\usepackage{enumitem}
\usepackage[polish]{babel}
\usepackage[T1]{fontenc}
\usepackage[utf8]{inputenc}
\usepackage[margin=1.9cm]{geometry}
\usepackage[edges]{forest}
\usepackage{listings}
\usepackage{multicol}
\usepackage{tgbonum}
\usepackage{xcolor}

\begin{document}
\large
\fontfamily{cmss}\selectfont

\title{Algorytmy i struktury danych (Sortowanie i kopce)}
\date{}
\maketitle

Przyjmując, że $t1[]={1,2,3,4,5,6,7}$ oraz $t2[]={7,6,5,4,3,2,1}$ i stosując algorytmy sortujące ściśle wg procedur z pliku sorty2020.cc i wykonaj polecenia:

\paragraph{Zadanie 1} Ile dokładnie porównań (między elementami  tablicy) wykona insertionSort(t2) a ile insertionSsort(t1)? \\

Posortowana: n-1 porównań. Nieposortowana: $\frac{n(n-1)}{2}$ porównań. \\
Dla t1 wykona 6 porównań. Dla t2 wykona 21 porównań.

\paragraph{Zadanie 2} Ile co najwyżej porównań (między elementami tablic) wykona procedura scalająca merge dwie tablice n-elementowe? \\


\paragraph{Zadanie 3} Jaka jest pesymistyczna złożoność czasowa procedury mergeSort? Odpowiedź uzasadnij. \\


Dla wersji rekurencyjnej i iteracyjnej: $O(nlogn)$ 
Porównujemy elementy i scalamy je w jedną tablicę. W każdym kroku wykonujemy $n$ porównań. W każdym kroku dzielimy tablicę na dwie części o połowie długości. W sumie wykonujemy $logn$ kroków. Złożoność czasowa zawsze jest $O(nlogn)$.

\paragraph{Zadanie 4} Ile co najwyżej porównań (między elementami tablicy) wykona procedura partition? \\


Przy jednym wywołaniu partition wykona się maksymalnie $n-1$ porównań. To przypadek, gdy nasze odwrócone elementy są najbliżej pivota lub musimy zamienić stronami każdy z elementó. W obu przypadkach musimy każdą wartość po stronie prawej i lewej porównać z pivotem.

\paragraph{Zadanie 5} Jak jest średnia a jaka pesymistyczna złożoność quickSort. Odpowiedź uzasadnij. \\


Średnia złożoność czasowa: $O(nlogn)$ \\
Pesymistyczna złożoność czasowa: $O(n^2)$ \\
Pesymistyczną złożoność osiągamy, gdy jako pivot zostanie wybrana wartość najmniejsza lub największa w tablicy. Złożoność kwadratowa wynika ze wzoru \\
$T(n) = n - 1 + T(n-1) -> T(n)=\frac{n(n-1)}{2}$

\paragraph{Zadanie 6} Jaka jest złożoność funkcji buildheap? Przeprowadź dowód - uzasadnij swoją odpowiedź.

\paragraph{Zadanie 7} 

\paragraph{Zadanie 8} 

\paragraph{Zadanie 9} 


\end{document}