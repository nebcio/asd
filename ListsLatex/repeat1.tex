\documentclass[18pt]{extarticle}

\usepackage{amsmath}
\usepackage{amssymb}
\usepackage{enumitem}
\usepackage[polish]{babel}
\usepackage[T1]{fontenc}
\usepackage[utf8]{inputenc}
\usepackage[margin=1.9cm]{geometry}
\usepackage[edges]{forest}
\usepackage{listings}
\usepackage{multicol}
\usepackage{tgbonum}
\usepackage{xcolor}

\begin{document}
\large
\fontfamily{cmss}\selectfont

\title{Algorytmy i struktury danych (Sortowanie i kopce)}
\date{}
\maketitle

Przyjmując, że $t1[]={1,2,3,4,5,6,7}$ oraz $t2[]={7,6,5,4,3,2,1}$ i stosując algorytmy sortujące ściśle wg procedur z pliku sorty2020.cc i wykonaj polecenia:

\paragraph{Zadanie 1} Ile dokładnie porównań (między elementami  tablicy) wykona insertionSort(t2) a ile insertionSsort(t1)? \\

Posortowana: n-1 porównań. Nieposortowana: $\frac{n(n-1)}{2}$ porównań. \\
Dla t1 wykona 6 porównań. Dla t2 wykona 21 porównań.

\paragraph{Zadanie 2} Ile co najwyżej porównań (między elementami tablic) wykona procedura scalająca merge dwie tablice n-elementowe? \\


W najgorszym przypadku będzie musiała wykonać $2n - 1$ porównań, gdzie lementy rosnące wystepują naprzemiennie, więc będziemy porównywać każdy element do momentu aż nie trafimy na większy i go nie wpiszemy do tablicy wyjściowej.

\paragraph{Zadanie 3} Jaka jest pesymistyczna złożoność czasowa procedury mergeSort? Odpowiedź uzasadnij. \\


Dla wersji rekurencyjnej i iteracyjnej: $O(nlogn)$ \\

Twierdzenie o rekurencji uniwersalnej: $T(n) = 2T(\frac{n}{2}) + n$,\\ gdzie liczba podproblemów to $a = 2$, rozmiar podproblemu to $b = 2$, i mamy n jako liczbę porównań.\\

$f(n)=\theta(n^{log_{2}{2}})=\theta(n)$\\

$T(n)=\theta(nlogn)$ \\

Porównujemy elementy i scalamy je w jedną tablicę. W każdym kroku wykonujemy $n$ porównań. W każdym kroku dzielimy tablicę na dwie części o połowie długości. W sumie wykonujemy $logn$ kroków, scalań. Złożoność czasowa zawsze jest $O(nlogn)$.

\paragraph{Zadanie 4} Ile co najwyżej porównań (między elementami tablicy) wykona procedura partition? \\


Przy jednym wywołaniu partition wykona się maksymalnie $n+1$ porównań. To przypadek, gdy nasze odwrócone elementy są najbliżej pivota lub musimy zamienić stronami każdy z elementów. W obu przypadkach musimy każdą wartość po stronie prawej i lewej porównać z pivotem. Mimalnie są zawsz 2 porównania.

\paragraph{Zadanie 5} Jak jest średnia a jaka pesymistyczna złożoność quickSort. Odpowiedź uzasadnij. \\


Średnia złożoność czasowa: $O(nlogn)$ \\

Pesymistyczna złożoność czasowa: $O(n^2)$ \\

Pesymistyczną złożoność osiągamy, gdy jako pivot zostanie wybrana wartość najmniejsza lub największa w tablicy. Złożoność kwadratowa wynika ze wzoru \\

$T(n) = n + 1 + T(n-1) -> T(n)=\frac{n(n-1)}{2}$ \\

czyli możemy posłużyć się sumą ciągu arytmetycznego.\\

$T(n)=O(n^2)$

\paragraph{Zadanie 6} Jaka jest złożoność funkcji buildheap? Przeprowadź dowód - uzasadnij swoją odpowiedź. \\

Złożoność buildheap wynosi $O(n)$.\\

Jeśli tablica ma n elementów to $\frac{n}{2}$ elementów jest liśćmi. Do porównań potrzebujemy węzła z dziećmi, więc ich nie mamy z czym porównywać. Pozosłe węzły możemy porównywać z dziećmi, a przy przesiewaniu z dalszymi potomkami. Stąd: \\

$2(\frac{n}{2} \cdot 0 + \frac{n}{4} \cdot 1 + \frac{n}{8} \cdot 2 + ...)$ \\

Mnożenie przez dwa wynika z sytuacji gdy porównujemy oboje dzieci z rodzicem.

\begin{align*}
    2 \sum_{i=1}^{h-1} \frac{n}{2^{i+1}} \cdot i                                                 &                                               \\
    2 (\frac{n}{4} \cdot 1 + \frac{n}{8} \cdot 2 + \frac{n}{16} \cdot 3 + \frac{n}{32} \cdot 4 + & \dots)                                        \\
    \frac{n}{2} (\frac{1}{1} + \frac{2}{2} + \frac{3}{4} + \frac{4}{8} +                         & \dots)                                        \\
    \frac{1}{1} +  \frac{1}{2} +  \frac{1}{4} +  \frac{1}{8} +                                   & \dots = \frac{\frac{1}{1}}{1-\frac{1}{2}} = 2 \\
    \frac{1}{2} +  \frac{1}{4} +  \frac{1}{8} +                                                  & \dots = 1                                     \\
    \frac{1}{4} +  \frac{1}{8} +                                                                 & \dots = 0.5                                   \\
    \frac{1}{8} +                                                                                & \dots = 0.25                                  \\
                                                                                                 & \dots
\end{align*}
\begin{gather*}
    \frac{n}{2} \cdot \frac{2}{1-\frac{1}{2}} = \frac{n}{2} \cdot 4 = 2n = O(n)
\end{gather*}

\paragraph{Zadanie 7} Ile dodatkowej pamięci wymaga posortowanie tablicy n-elementowej za pomocą algorytmu: \\
\begin{enumerate}[label=\alph*.]
    \item mergesort, złożoność pamięciowa 
    \item quicksort
    \item heapsort, złożoność pamięciowa O(n)\\
            nie potrzebujemy dodatkowej pamięci -> O(1), ponieważ sortujemy w miejscu
    \item insertionsort
    \item countingsort
    \item bucketsort
    \item radixsort
    W punktach (e), (f), (g) zakładamy, że ilość kubełków jest m, a liczby do posortowania mają nie więcej niż k cyfr.
\end{enumerate}

\paragraph{Zadanie 8} Jaka jest średnia a jaka pesymistyczna złożoność czasowa algorytmu: \\
\begin{enumerate}[label=\alph*]
    \item mergesort
    \item quicksort
    \item heapsort
    \item insertionsort
    \item countingsort
    \item bucketsort
    \item radixsort
    W punktach (e), (f), (g) zakładamy, że ilość kubełków jest m, a liczby do posortowania mają nie więcej niż k cyfr.
\end{enumerate}

\paragraph{Zadanie 9} Udowodnij, że wysokość (ilość poziomów na których występują węzły) kopca n-elementowego wynosi $\lfloor log_2 n \rfloor + 1$ \\


\paragraph{Zadanie 10}


\paragraph{Zadanie 11} Czy ciąg {\textcolor{green}{23}, \textcolor{green}{17}, 14, \textcolor{green}{6}, 13, 10, 1, \textcolor{green}{5}, \textcolor{red}{7}, 12} jest kopcem? \\

Nie jest to kopiec ze względu na to że 7 jest większa od swojego rodzica, którym jest 6. \\ 

\end{document}