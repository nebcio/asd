\documentclass[18pt]{extarticle}

\usepackage{amsmath}
\usepackage{amssymb}
\usepackage{enumitem}
\usepackage[polish]{babel}
\usepackage[T1]{fontenc}
\usepackage[utf8]{inputenc}
\usepackage[margin=1.9cm]{geometry}
\usepackage[edges]{forest}
\usepackage{listings}
\usepackage{multicol}
\usepackage{tgbonum}
\usepackage{xcolor}

\begin{document}
\large
\fontfamily{cmss}\selectfont

\title{Algorytmy i struktury danych (Sortowanie i kopce)}
\date{}
\maketitle

Przyjmując, że $t1[]={1,2,3,4,5,6,7}$ oraz $t2[]={7,6,5,4,3,2,1}$ i stosując algorytmy sortujące ściśle wg procedur z pliku sorty2020.cc i wykonaj polecenia:

\paragraph{Zadanie 1} Ile dokładnie porównań (między elementami  tablicy) wykona insertionSort(t2) a ile insertionSsort(t1)? \\

Posortowana: n-1 porównań. Nieposortowana: $\frac{n(n-1)}{2}$ porównań. \\
Dla t1 wykona 6 porównań. Dla t2 wykona 21 porównań.

\paragraph{Zadanie 2} Ile co najwyżej porównań (między elementami tablic) wykona procedura scalająca merge dwie tablice n-elementowe? \\


W najgorszym przypadku będzie musiała wykonać $2n - 1$ porównań, gdzie lementy rosnące wystepują naprzemiennie, więc będziemy porównywać każdy element do momentu aż nie trafimy na większy i go nie wpiszemy do tablicy wyjściowej.

\paragraph{Zadanie 3} Jaka jest pesymistyczna złożoność czasowa procedury mergeSort? Odpowiedź uzasadnij. \\


Dla wersji rekurencyjnej i iteracyjnej: $O(nlogn)$ \\

Twierdzenie o rekurencji uniwersalnej: $T(n) = 2T(\frac{n}{2}) + n$,\\ gdzie liczba podproblemów to $a = 2$, rozmiar podproblemu to $b = 2$, i mamy n jako liczbę porównań.\\

$f(n)=\theta(n^{log_{2}{2}})=\theta(n)$\\

$T(n)=\theta(nlogn)$ \\

Porównujemy elementy i scalamy je w jedną tablicę. W każdym kroku wykonujemy $n$ porównań. W każdym kroku dzielimy tablicę na dwie części o połowie długości. W sumie wykonujemy $logn$ kroków, scalań. Złożoność czasowa zawsze jest $O(nlogn)$.

\paragraph{Zadanie 4} Ile co najwyżej porównań (między elementami tablicy) wykona procedura partition? \\


Przy jednym wywołaniu partition wykona się maksymalnie $n+1$ porównań. To przypadek, gdy nasze odwrócone elementy są najbliżej pivota lub musimy zamienić stronami każdy z elementów. W obu przypadkach musimy każdą wartość po stronie prawej i lewej porównać z pivotem. Mimalnie są zawsz 2 porównania.

\paragraph{Zadanie 5} Jak jest średnia a jaka pesymistyczna złożoność quickSort. Odpowiedź uzasadnij. \\


Średnia złożoność czasowa: $O(nlogn)$ \\

Pesymistyczna złożoność czasowa: $O(n^2)$ \\

Pesymistyczną złożoność osiągamy, gdy jako pivot zostanie wybrana wartość najmniejsza lub największa w tablicy. Złożoność kwadratowa wynika ze wzoru \\

$T(n) = n + 1 + T(n-1) -> T(n)=\frac{n(n-1)}{2}$ \\

czyli możemy posłużyć się sumą ciągu arytmetycznego.\\

$T(n)=O(n^2)$

\paragraph{Zadanie 6} Jaka jest złożoność funkcji buildheap? Przeprowadź dowód - uzasadnij swoją odpowiedź. \\

Złożoność buildheap wynosi $O(n)$.\\

Jeśli tablica ma n elementów to $\frac{n}{2}$ elementów jest liśćmi. Do porównań potrzebujemy węzła z dziećmi, więc ich nie mamy z czym porównywać. Pozosłe węzły możemy porównywać z dziećmi, a przy przesiewaniu z dalszymi potomkami. Stąd: \\

$2(\frac{n}{2} \cdot 0 + \frac{n}{4} \cdot 1 + \frac{n}{8} \cdot 2 + ...)$ \\

Mnożenie przez dwa wynika z sytuacji gdy porównujemy oboje dzieci z rodzicem.

\begin{align*}
    2 \sum_{i=1}^{h-1} \frac{n}{2^{i+1}} \cdot i                                                 &                                               \\
    2 (\frac{n}{4} \cdot 1 + \frac{n}{8} \cdot 2 + \frac{n}{16} \cdot 3 + \frac{n}{32} \cdot 4 + & \dots)                                        \\
    \frac{n}{2} (\frac{1}{1} + \frac{2}{2} + \frac{3}{4} + \frac{4}{8} +                         & \dots)                                        \\
    \frac{1}{1} +  \frac{1}{2} +  \frac{1}{4} +  \frac{1}{8} +                                   & \dots = \frac{\frac{1}{1}}{1-\frac{1}{2}} = 2 \\
    \frac{1}{2} +  \frac{1}{4} +  \frac{1}{8} +                                                  & \dots = 1                                     \\
    \frac{1}{4} +  \frac{1}{8} +                                                                 & \dots = 0.5                                   \\
    \frac{1}{8} +                                                                                & \dots = 0.25                                  \\
                                                                                                 & \dots
\end{align*}
\begin{gather*}
    \frac{n}{2} \cdot \frac{2}{1-\frac{1}{2}} = \frac{n}{2} \cdot 4 = 2n = O(n)
\end{gather*}

\paragraph{Zadanie 7} Ile dodatkowej pamięci wymaga posortowanie tablicy n-elementowej za pomocą algorytmu:
\begin{enumerate}[label=\alph*.]
    \item mergesort, złożoność pamięciowa O(n)
            dodatkowa pamięć O(n)
    \item quicksort, złożoność pamięciowa O(n)
            dodatkowa pamięć O(n)
    \item heapsort, złożoność pamięciowa O(n)\\
            nie potrzebujemy dodatkowej pamięci -> O(1), ponieważ sortujemy w miejscu
    \item insertionsort, złożoność pamięciowa O(n)\\
            dodatkowa pamięć O(1)
    \item countingsort, złożoność pamięciowa O(n)\\
            dodatkowa pamięć O(n)
    \item bucketsort, złożoność pamięciowa O(n)\\
            dodatkowa pamięć O(n + m)
    \item radixsort, złożoność pamięciowa O(n)\\
            dodatkowa pamięć O(n + k)\\

    W punktach (e), (f), (g) zakładamy, że ilość kubełków jest m, a liczby do posortowania mają nie więcej niż k cyfr.
\end{enumerate}

\paragraph{Zadanie 8} Jaka jest średnia a jaka pesymistyczna złożoność czasowa algorytmu:
\begin{enumerate}[label=\alph*]
    \item mergesort nlog n
    \item quicksort nlog n
    \item heapsort nlog n  $n^2$
    \item insertionsort $n^2 n^2$
    \item countingsort n + k n + k
    \item bucketsort n + m n + m
    \item radixsort nk nk
    W punktach (e), (f), (g) zakładamy, że ilość kubełków jest m, a liczby do posortowania mają nie więcej niż k cyfr.
\end{enumerate}

\paragraph{Zadanie 9} Udowodnij, że wysokość (ilość poziomów na których występują węzły) kopca n-elementowego wynosi $\lfloor log_2 n \rfloor + 1$ \\

\begin{center}
    \begin{forest}
        for tree={circle, draw, minimum size=3ex, inner sep=1pt, s sep=7mm, anchor=south, fill=green!50}
        [7[6[4][3]][5[2][1]]]
    \end{forest} \\
    i = ostatnie dziecko z dziećmi = $n/2-1$ \\
    $i = 0$\\
    $2(2(2 * 0+1)+1)+1... < n$
\end{center}
\begin{gather*}
    \text{Maksymalna ilość węzłów w kopcu o wysokości $h$:} \\
    n(h) = 2^h - 1 \\
    n(h-1) = 2^{h - 1} - 1 \\
    \text{Minimalna ilość węzłów w kopcu o wysokości $h$:} \\
    n(h-1) + 1 = 2^{h - 1} - 1 + 1 = 2^{h - 1} \\
    \text{Ilość węzłów w kopcu o wysokości $h$:} \\
    2^{h - 1} \leq n < 2^h \\
    h-1 \leq \log_2 n < h \\
    h \leq \log_2 n + 1 < h + 1 \\
    \lfloor\log_2 n\rfloor + 1
\end{gather*}

\paragraph{Zadanie 10} Który element tablicy t jest (a) lewym dzieckiem (b) prawym dzieckim (c) ojcem, elementu t[i] w procedurze heapsort?

\begin{gather*}
    (a) t[2i+1] \\
    (b) t[2i+2] \\
    (c) t[\lfloor\frac{i-1}{2}\rfloor]
\end{gather*}


\paragraph{Zadanie 11} Czy ciąg {\textcolor{green}{23}, \textcolor{green}{17}, 14, \textcolor{green}{6}, 13, 10, 1, \textcolor{green}{5}, \textcolor{red}{7}, 12} jest kopcem? \\

Nie jest to kopiec ze względu na to że 7 jest większa od swojego rodzica, którym jest 6. \\ 

\paragraph{Zadanie 12} Zilustruj działanie procedury buildheap dla ciągu {5,3,17,10,84,19,6,22,9}. Narysuj na kartce wygląd tablicy/kopca po każdym wywołaniu procedury przesiej.

\paragraph{Zadanie 13} Zasymuluj działanie polifazowego mergesorta dla tablicy {9,22,6,19,14,10,17,3,5}. Na każdym etapie sortowania scala się sąsiadujące listy rosnące.


\begin{align*}
    &{9,22 || 6,19 || 14 || 10,17 || 3,5} \\
    &{6,9,19,22 || 10,14,17 || 3,5} \\
    &{6,9,10,14,17,19,22 || 3,5} \\
    &{3,5,6,9,10,14,17,19,22}
\end{align*}

\paragraph{Zadanie 14} Zasymuluj działanie mergesort(t2).
\begin{align*}
    t2[]=&{7,6,5,4,3,2,1} \\
    &{7,6,5 || 4,3,2 || 1} \\
    &{6,7 || 3,4 || 1 || 2,5} \\
    &{3,4,6,7 || 1 || 2,5} \\
    &{1 || 2,3,4,5,6,7} \\
    &{1,2,3,4,5,6,7}
\end{align*}


\paragraph{Zadanie 15} Zasymuluj działanie partition(t2,7).

\begin{align*}
    t2[]=&{7,6,5,4,3,2,1} \\
    x=4, k=0, n=6, swap =>
    &{1,6,5,4,3,2,7} \\
    x=4, k=1, n=5, swap =>
    &{1,2,5,4,3,6,7} \\
    x=4, k=2, n=4, swap =>
    &{1,2,3,4,5,6,7} \\
    return\text{ }3;
\end{align*}


\paragraph{Zadanie 16} Zasymuluj działanie partition(t2,7) w przypadku gdyby piwotem zamiast t[n/2] było t[0]

\begin{align*}
    t2[]=&{7,6,5,4,3,2,1} \\
    x=7, k=6, n=6& \\
    return\text{ }6;
\end{align*}

\paragraph{Zadanie 17} Patrz 5.

\paragraph{Zadanie 18} Napisz wzór na numer kubełka, do którego należy wrzucić liczbę x w sortowaniu kubełkowym, jeśli kubełków jest n, a elementy tablicy mieszczą się przedziale (a, b). Numeracja zaczyna się od 0

\paragraph{Zadanie 19} Jak obliczyć k-tą od końca cyfrę w liczby x? Jak obliczyć ilość cyfr liczby x? Przyjmujemy układ dziesiętny. Jak wyniki zmienią się w układzie pozycyjnym o 1000 cyfr?

\end{document}