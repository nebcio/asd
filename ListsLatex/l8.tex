\documentclass[18pt]{extarticle}

\usepackage{amsmath}
\usepackage{amssymb}
\usepackage{enumitem}
\usepackage[polish]{babel}
\usepackage[T1]{fontenc}
\usepackage[utf8]{inputenc}
\usepackage[margin=1.9cm]{geometry}
\usepackage[edges]{forest}
\usepackage{listings}
\usepackage{multicol}
\usepackage{tgbonum}
\usepackage{xcolor}
\usetikzlibrary{arrows.meta}

\begin{document}
\large
\fontfamily{cmss}\selectfont

\title{Algorytmy i struktury danych (Lista 8)}
\date{}
\maketitle

\paragraph{Zadanie 1} 
\begin{enumerate}
    \item Czym różni się haszowanie łańcuchowe od otwartego? \\
        Łańcuchowe: elementy o danym indeksie są przechowywane w liście pod danym indeksemw  tablicy (czasem drzewie) \\
        Otwarte: elementy trafiają bezpośrednio do tablicy, ale pod obliczany indeks
    \item Czym różnią się dwie wersje haszowania otwartego: haszowanie liniowe i haszowanie podwójne? \\
        Liniowe -> pierwsza wolna komórka \\
        Podwójne -> po wykryciu kolizji przy h() sprawdza h() + h2()[wielokrotności h2 dodaje] mod n \\
    \item Dla tablicy z haszowaniem podwójnym o rozmiarze m = 11 i funkcjach haszujących: h1(x) = x mod 11 oraz h2(x) = x mod 10 + 1 wyznacz ciąg kontrolny dla liczby 23. Jak wyglądałby ten ciąg w przypadku haszowania liniowego? \\
        Podwójne: 1, 5, 9, 2, 6, 10, 3, 7, 0, 4, 8 \\
        Liniowe:  1, 2, 3, 4, 5, 6, 7, 8, 9, 10, 0 \\
\end{enumerate}

\paragraph{Zadanie 2} (2pkt) Porównaj jaka będzie łączna liczba kolizji, gdy do tablicy z poprzedniego zadania wstawimy kolejno liczby: 22, 66, 44, 23, 35, używając: 
\begin{enumerate}[label=(\alph*)]
    \item haszowania liniowego: 7
    \item haszowania dwukrotnego: 2
    \item haszowania łańcuchowego: 2
\end{enumerate}
Następnie w każdym z wariantów sprawdź, jaka będzie łączna liczba porównań kluczy, gdy w gotowej tablicy wywołamy kolejno procedurę FIND (a) dla każdego elementu obecnego w tablicy, (b) dla elementów: 24 i 34, których nie ma w tablicy.
\begin{enumerate}[label=(\alph*)]
    \item haszowania liniowego:     a] 12 b] 7
    \item haszowania dwukrotnego:   a] 7 b] 4
    \item haszowania łańcuchowego:  a] 8 b] 2
\end{enumerate}

\pagebreak
\paragraph{Zadanie 3} Narysuj przykładowe kopce dwumianowe o 12, 14, 15 i 16 węzłach. Na rysunku uwzględnij wartości kluczy oraz stopnie węzłów. W kopcu 12-elementowym zaznacz dodatkowo strzałki (najlepiej w różnych kolorach) przedstawiające wskaźniki na ojca, syna i brata.
\begin{center}
    \begin{forest}
        [, phantom, for tree={circle, draw, minimum size=3ex, inner sep=1pt, s sep=5mm, edge=Latex-, calign=last},
            [12[9[6]][11]]{\draw[-Latex] () to node{} (1);}
                [10, name=1[8[2[1]][7]][4[3]][5]]
        ]
    \end{forest}
    \qquad
    \begin{forest}
        [, phantom, for tree={circle, draw, minimum size=3ex, inner sep=1pt, s sep=5mm, edge=Latex-, calign=last},
            [14[13]]{\draw[-Latex] () to node{} (1);}
                [12, name=1[9[6]][11]]{\draw[-Latex] () to node{} (2);}
                [10, name=2[8[2[1]][7]][4[3]][5]]
        ]
    \end{forest}
\end{center}
\begin{center}
    \begin{forest}
        [, phantom, for tree={circle, draw, minimum size=3ex, inner sep=1pt, s sep=5mm, edge=Latex-, calign=last},
            [15]{\draw[-Latex] () to node{} (1);}
                [14, name=1[13]]{\draw[-Latex] () to node{} (2);}
                [12, name=2[9[6]][11]]{\draw[-Latex] () to node{} (3);}
                [10, name=3[8[2[1]][7]][4[3]][5]]
        ]
    \end{forest}
    \qquad
    \begin{forest}
        [, phantom, for tree={circle, draw, minimum size=3ex, inner sep=1pt, s sep=5mm, edge=Latex-, calign=last},
            [16[10[8[2[1]][7]][4[3]][5]][12[9[6]][11]][14[13]][15]]
        ]
    \end{forest}
\end{center}

\paragraph{Zadanie 4} Zilustruj działanie operacji UNION łączącej kopiec dwumianowy o 14 węzłach z kopcem dwumianowym o 11 węzłach. Przyjmij dowolne wartości kluczy spełniające warunek kopca.
\begin{center}
    \begin{forest}
        [, phantom, for tree={circle, draw, minimum size=3ex, inner sep=1pt, s sep=5mm, edge=Latex-, calign=last},
            [14[13]]{\draw[-Latex] () to node{} (1);}
                [12, name=1[10[9]][11]]{\draw[-Latex] () to node{} (2);}
                [8, name=2[4[2[1]][3]][6[5]][7]]
        ]
    \end{forest}
    \qquad
    \begin{forest}
        [, phantom, for tree={circle, draw, minimum size=3ex, inner sep=1pt, s sep=5mm, edge=Latex-, calign=last},
            [25]{\draw[-Latex] () to node{} (1);}
                [24, name=1[23]]{\draw[-Latex] () to node{} (2);}
                [22, name=2[18[16[15]][17]][20[19]][21]]
        ]
    \end{forest}
\end{center}
\begin{center}
    \begin{forest}
        [, phantom, for tree={circle, draw, minimum size=3ex, inner sep=1pt, s sep=5mm, edge=Latex-, calign=last},
            [25]{\draw[-Latex] () to node{} (1);}
                [14, name=1[13]]{\draw[-Latex] () to node{} (2);}
                [24, name=2[23]]{\draw[-Latex] () to node{} (3);}
                [12, name=3[10[9]][11]]{\draw[-Latex] () to node{} (4);}
                [8,  name=4[4[2[1]][3]][6[5]][7]]{\draw[-Latex] () to node{} (5);}
                [22, name=5[18[16[15]][17]][20[19]][21]]
        ]
    \end{forest}
\end{center}
\begin{center}
    \begin{forest}
        [, phantom, for tree={circle, draw, minimum size=3ex, inner sep=1pt, s sep=5mm, edge=Latex-, calign=last},
            [25]{\draw[-Latex] () to node{} (1);}
                [24, name=1[14[13]][23]]{\draw[-Latex] () to node{} (2);}
                [12, name=2[10[9]][11]]{\draw[-Latex] () to node{} (3);}
                [8,  name=3[4[2[1]][3]][6[5]][7]]{\draw[-Latex] () to node{} (4);}
                [22, name=4[18[16[15]][17]][20[19]][21]]
        ]
    \end{forest}
\end{center}
\begin{center}
    \begin{forest}
        [, phantom, for tree={circle, draw, minimum size=3ex, inner sep=1pt, s sep=5mm, edge=Latex-, calign=last},
            [25]{\draw[-Latex] () to node{} (1);}
                [24, name=1[12[10[9]][11]][14[13]][23]]{\draw[-Latex] () to node{} (2);}
                [8,  name=2[4[2[1]][3]][6[5]][7]]{\draw[-Latex] () to node{} (3);}
                [22, name=3[18[16[15]][17]][20[19]][21]]
        ]
    \end{forest}
\end{center}
\begin{center}
    \begin{forest}
        [, phantom, for tree={circle, draw, minimum size=3ex, inner sep=1pt, s sep=5mm, edge=Latex-, calign=last},
            [25]{\draw[-Latex] () to node{} (1);}
                [24, name=1[12[10[9]][11]][14[13]][23]]{\draw[-Latex] () to node{} (2);}
                [22, name=2[8[4[2[1]][3]][6[5]][7]][18[16[15]][17]][20[19]][21]]
        ]
    \end{forest}
\end{center}

\paragraph{Zadanie 6} Napisz funkcję \verb`int ile_drzew_w_kopcu(int n)` wyliczającą ile jest drzew dwumianowych w kopcu dwumianowym zawierającym $n$ kluczy.
\begin{lstlisting}
    int ile_drzew_w_kopcu(int n) {
        int i = 0;  // zlicza 1 w bitowej reprezentacji n
        while (n > 0) {
            if (n % 2 == 1) ++i;
            n /= 2;
        }
        return i;
    }
\end{lstlisting}

\paragraph{Zadanie 7}
\begin{enumerate}[label=(\alph*)]
    \item Do pustego kopca dwumianowego wstaw (INSERT) kolejno: 1, 12, 3, 14, 5, 16, 7, 20, 25, 13, 8
          \begin{center}
              \begin{forest}
                  [, phantom, for tree={circle, draw, minimum size=3ex, inner sep=1pt, s sep=5mm, edge=Latex-, calign=last},
                      [8]{\draw[-Latex] () to node{} (1);}
                          [25, name=1[13]]{\draw[-Latex] () to node{} (2);}
                          [20, name=2[14[12[1]][3]][16[5]][7]]
                  ]
              \end{forest}
          \end{center}
    \item Dla otrzymanego kopca dwukrotnie wykonaj operację GETMAX (kradzież max).
          \begin{center}
              \begin{forest}
                  [, phantom, for tree={circle, draw, minimum size=3ex, inner sep=1pt, s sep=5mm, edge=Latex-, calign=last},
                      [13[8]]{\draw[-Latex] () to node{} (1);}
                          [20, name=1[14[12[1]][3]][16[5]][7]]
                  ]
              \end{forest}
              \qquad
              \begin{forest}
                  [, phantom, for tree={circle, draw, minimum size=3ex, inner sep=1pt, s sep=5mm, edge=Latex-, calign=last},
                      [7]{\draw[-Latex] () to node{} (1);}
                      [16, name=1[14[12[1]][3]][13[8]][5]]
                  ]
              \end{forest}
          \end{center}
\end{enumerate}

\end{document}