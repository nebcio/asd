\documentclass[18pt]{extarticle}

\usepackage{amsmath}
\usepackage{amssymb}
\usepackage{enumitem}
\usepackage[polish]{babel}
\usepackage[T1]{fontenc}
\usepackage[utf8]{inputenc}
\usepackage[margin=1.9cm]{geometry}
\usepackage[edges]{forest}
\usepackage{listings}
\usepackage{multicol}
\usepackage{tgbonum}
\usepackage{xcolor}

\begin{document}
\large
\fontfamily{cmss}\selectfont

\title{Algorytmy i struktury danych (Lista 7)}
\date{}
\maketitle

\text{ Jakie informacje przechowujemy w węźle B-drzewa? Podaj definicję B-drzewa.} \\

\begin{lstlisting}
    struct node {
        int key;
        node* left;
        node* right;
        node(int k, node* l, node* r):key(k),left(l),right(r){}
    };
\end{lstlisting}
\paragraph{Zadanie 1} Warunki klucza drzewa BST
\begin{enumerate}
    \item Lewa strona zawiera wartości mniejsze od rodzica
    \item Prawa większe lub równe
    \item Llucz z wartością
\end{enumerate}

\paragraph{Zadanie 2} Napisz procedurę \verb|node* find(node* tree, int x)|, która zwraca wskaźnik na węzeł zawierający x, lub NULL, jeśli nie ma takiego węzła. \\

\begin{lstlisting}
    node* find(node* tree, int x) {
        while (tree != null && tree->key != x)
            if (x < tree->key) tree = tree->left
            else tree = tree->right
    }
\end{lstlisting}

\paragraph{Zadanie 3} Napisz procedurę \verb|void insert(node*& tree, int x)| (dodaje do drzewa tree klucz x). \\

\begin{lstlisting}
    node* insert(node*& tree, int x) {
        node **node = &tree;
        while (*tree != null)
            if (x < (**node).key) node = &(**node).left
            else node = &(**node).right

        *node = new node(x)
    }
\end{lstlisting}

\paragraph{Zadanie 4} Drzewo BST o różnych kluczach można odtworzyć z listy par kluczWezła:kluczOjca. 
(a) Narysuj drzewo BST reprezentowane przez listę par: 1:2, 2:4, 3:2, 4:5, 6:7, 7:9, 8:7, 9:5.
\begin{center}
    \begin{forest}
        [, phantom, for tree={circle, minimum size=3ex, inner sep=1pt, s sep=5mm, l sep=0mm, l=0mm, anchor=south, fill=black, text=white},
        [5[4[2[1][3]][, phantom]][9[7[6][8]][, phantom]]]
        ]
    \end{forest}
\end{center}
(b) wypisz jego klucze w porządku: INORDER, (c) PREORDER, (d) POSTORDER
\begin{enumerate}
    \item INORDER:  1 2 3 4 5 6 7 8 9
    \item PREORDER: 5 4 2 1 3 9 7 6 8
    \item POSTORDER:1 3 2 4 6 8 7 9 5
\end{enumerate}

\paragraph{Zadanie 5} Napisz procedurę \verb|void wypisz(node *tree, int order=0)|, która wypisuje klucze drzewa tree w porządku inorder gdy order=0, preorder gdy order=1, postorder gdy order=2 \\


\begin{lstlisting}
    void wypisz(node *tree, int order = 0) {
        if (tree == nullptr) return;
        
        if (order == 1) std::cout << tree->key;
        wypisz(tree->left, order);
        if (order == 0) std::cout << tree->key;
        wypisz(tree->right, order);
        if (order == 2) std::cout << tree->key;
    }
\end{lstlisting}

\paragraph{Zadanie 6} Jakie informacje przechowujemy w węźle drzewa czerwono-czarnego? Podaj definicję drzewa czerwono czarnego. Zadeklaruj strukturę RBnode tak, by dziedziczyła z node. Czy można dla niej użyć funkcji napisanych w zadaniach 2, 3 i 5? \\


RBT: find tak samo (5), wypisz/print tak samo (5), insert nie (3) \\

Oprócz lewego/prawego dziecka kolor węzła (czerwony lub czarny). 
Logika/definicja: drzewo czerwono-czarne jest drzewem BST, w którym każdy węzeł ma dodatkowo kolor (czerwony lub czarny). Węzły czerwone mają 2 czarnych dzieci. Korzeń jest czarny. Każda ścieżka od korzenia do liścia ma tę samą liczbę czarnych węzłów. Każdy liść jest czarny (wartownik) a czerwony węzeł nie może mieć czerwonego dziecka. \\

\begin{lstlisting}
struct BSTnode:
    int key;	 
    BSTnode *left;	 
    BSTnode *right;
end

struct RBTnode : BSTnode:
    bool color;	 
    RBT* parent;
end
\end{lstlisting}

\paragraph{Zadanie 7} Uzasadnij posługując się rysunkiem i opisem, że operacje na drzewie czerwono-czarnym (rotacja i przekolorowanie) nie zmieniają ilości czarnych węzłów, na żadnej ścieżce od korzenia
do liścia

\paragraph{Zadanie 8} W poniższym drzewie czerwono-czarnym (czarne węzły oznaczono nawiasem kwadratowym), usuń 1, dodaj do wyściowego 10:

\begin{multicols}{2}
    \begin{lstlisting}
            [5] 
       (3)     [11] 
     [1] [4]  (9) 
    
    
             [5] 
         (3)     [11] 
     [1]   [4]  (9) 
                 (10) 
     
             [5] 
         (3)      [11] 
     [1]   [4]   (10) 
                (9) 
     
             [5] 
         (3)      [10] 
     [1]   [4]   (9) (11)
     
             [5] 
         (3)      (10) 
     [1]   [4]   [9] [11]
                 
    \end{lstlisting}
    
    \columnbreak
    \begin{lstlisting}
             [5] 
         (3)      (10) 
          [4]   [9] [11]
    
             [5] 
         [3]      (10) 
          (4)   [9] [11]
    \end{lstlisting}
    \end{multicols}

\paragraph{Zadanie 9} Jakie informacje przechowujemy w węźle B-drzewa? Podaj definicję B-drzewa

\begin{lstlisting}
    struct BTnode:
        int t;              // minimum degree
        int n;              // current number of keys
        int* keys;          // array of keys in non-decreasing order
        bool leaf;          // is it a leaf?
        BTnode** children;
    end
\end{lstlisting}
Inne założenia (oprócz zawartych w komentarzach):
\begin{enumerate}
    \item Węzeł wewnętrzny zawiera $n+1$ wskaźników do synów.
    \item Klucze rozdzielają dzieci na przedziały (n+1).
    \item Każdy węzeł różny od korzenia musi mieć co najmniej $t-1$ kluczy i co najwyżej $2t-1$ kluczy. (korzeń może mieć od $1$ do $2t-1$ kluczy)
    \item Wszystkie liście leżą na tej samej wysokości równym $h$.
\end{enumerate}

\paragraph{Zadanie 10} Narysuj B-drzewo o t = 3 zawierające dokładnie 17 kluczy na trzech poziomach: korzeń jego dzieci i wnuki. Następnie usuń z tego drzewa korzeń.


\begin{center}
    \begin{tikzpicture}
        \tikzstyle{bplus}=[rectangle split, rectangle split horizontal, rectangle split ignore empty parts, draw]
        \tikzstyle{every node}=[bplus]
        \tikzstyle{level 1}=[sibling distance=60mm]
        \tikzstyle{level 2}=[sibling distance=15mm]
        \node {9} [->]
        child {node {3 \nodepart{two} 6}
            child {node {1 \nodepart{two} 2}}
            child {node {4 \nodepart{two} 5}}
            child {node {7 \nodepart{two} 8}} 
        }
        child {node {12 \nodepart{two} 15}
            child {node {10 \nodepart{two} 11}}
            child {node {13 \nodepart{two} 14}}
            child {node {16 \nodepart{two} 17}}
        };
    \end{tikzpicture} \\ [5mm]
    Dzieci korzenia mają minimalną liczbę kluczy, więc robimy po prostu unsplit. \\ [5mm] 
    \begin{tikzpicture}
        \tikzstyle{bplus}=[rectangle split, rectangle split horizontal, rectangle split ignore empty parts, draw]
        \tikzstyle{every node}=[bplus]
        \tikzstyle{level 1}=[sibling distance=30mm]
        \tikzstyle{level 2}=[sibling distance=15mm]
        \node [rectangle split parts=5]{3 \nodepart{two} 6 \nodepart{three} 9 \nodepart{four} 12 \nodepart{five} 15} [->]
        child {node {1 \nodepart{two} 2}}
        child {node {4 \nodepart{two} 5}}
        child {node {7 \nodepart{two} 8}}
        child {node {10 \nodepart{two} 11}}
        child {node {12 \nodepart{two} 13 \nodepart{three} 14}}
        child {node {16 \nodepart{two} 17}};
    \end{tikzpicture} \\ [5mm]
    Teraz możemy usunąć 9. Jednak nie możemy zastąpić jej najmniejszym kluczem prawego dziecka, dlatego robimy unsplit dzieci. \\ [5mm]
    \begin{tikzpicture}
        \tikzstyle{bplus}=[rectangle split, rectangle split horizontal, rectangle split ignore empty parts, draw]
        \tikzstyle{every node}=[bplus]
        \tikzstyle{level 1}=[sibling distance=30mm]
        \tikzstyle{level 2}=[sibling distance=15mm]
        \node [rectangle split parts=5]{3 \nodepart{two} 6 \nodepart{three}  12 \nodepart{five} 15} [->]
        child {node {1 \nodepart{two} 2}}
        child {node {4 \nodepart{two} 5}}
        child {node {7 \nodepart{two} 8 \nodepart{three} 11 \nodepart{four} 10}}
        child {node {13 \nodepart{two} 14}}
        child {node {16 \nodepart{two} 17}};
    \end{tikzpicture}
\end{center}

\paragraph{Zadanie 11} Podano na rysunku B-drzewo o $t=2$:
\begin{center}
    \begin{tikzpicture}
        \tikzstyle{bplus}=[rectangle split, rectangle split horizontal, rectangle split ignore empty parts, draw]
        \tikzstyle{every node}=[bplus]
        \tikzstyle{level 1}=[sibling distance=60mm]
        \tikzstyle{level 2}=[sibling distance=15mm]
        \node {9} [->]
        child {node {7}
                child {node {6}}
                child {node {8}}
            }
        child {node {11 \nodepart{two} 14 \nodepart{three} 19}
                child {node {10}}
                child {node {12 \nodepart{two} 13}}
                child[sibling distance=25mm] {node {15 \nodepart{two} 16 \nodepart{three} 17}}
                child[sibling distance=20mm] {node {20}}
            };
    \end{tikzpicture}
    \begin{enumerate}[label=-]
        \item usuń z tego drzewa 7.
        \item do drzewa widocznego powyżej dodaj 18.
    \end{enumerate}
    Nie wchodzimy do dziecka o minimalnej liczbie ani maksymalnej liczbie węzłów. Dlatego od razu pożyczamy 11 od pełnego węzła i robimy z niej nowy korzeń. \\
    \begin{tikzpicture}
        \tikzstyle{bplus}=[rectangle split, rectangle split horizontal, rectangle split ignore empty parts, draw]
        \tikzstyle{every node}=[bplus]
        \tikzstyle{level 1}=[sibling distance=60mm]
        \tikzstyle{level 2}=[sibling distance=15mm]
        \node {11} [->]
        child {node {9}
            child {node {6 \nodepart{two} 8}}
            child {node {10}}
        }
        child {node {14 \nodepart{three} 19}
            child[sibling distance=25mm] {node {12 \nodepart{two} 13}}
            child {node {15 \nodepart{two} 16 \nodepart{three} 17}}
            child[sibling distance=25mm] {node {20}}
        };
    \end{tikzpicture} \\ [5mm]
    Dodanie 18: \\ [5mm]
    \begin{tikzpicture}
        \tikzstyle{bplus}=[rectangle split, rectangle split horizontal, rectangle split ignore empty parts, draw]
        \tikzstyle{every node}=[bplus]
        \tikzstyle{level 1}=[sibling distance=60mm]
        \tikzstyle{level 2}=[sibling distance=15mm]
        \node {9} [->]
        child {node {7}
            child {node {6}}
            child {node {8}}
        }
        child {node {11 \nodepart{two} 14 \nodepart{three} 19}
            child {node {10}}
            child {node {12 \nodepart{two} 13}}
            child[sibling distance=25mm] {node {15 \nodepart{two} 16 \nodepart{three} 17}}
            child[sibling distance=20mm] {node {20}}
        };
    \end{tikzpicture} \\ [5mm]
    Analogicznie jak wcześniej nie wchodzimy do pełnego węzła, tylko od razu robimy split i 14 idzie do góry. \\ [5mm]
    \begin{tikzpicture}
        \tikzstyle{bplus}=[rectangle split, rectangle split horizontal, rectangle split ignore empty parts, draw]
        \tikzstyle{every node}=[bplus]
        \tikzstyle{level 1}=[sibling distance=60mm]
        \tikzstyle{level 2}=[sibling distance=15mm]
        \node {9 \nodepart{two} 14} [->]
        child {node {7}
            child {node {6}}
            child {node {8}}
        }
        child {node {11}
            child {node {10}}
            child {node {12 \nodepart{two} 13}}
        }
        child {node {19}
            child[sibling distance=25mm] {node {15 \nodepart{two} 16 \nodepart{three} 17}}
            child[sibling distance=20mm] {node {20}}
        };
    \end{tikzpicture} \\ [5mm]
    Widząc znowu pełny węzeł robimy split i 16 idzie do rodzica. \\ [5mm]
    \begin{tikzpicture}
        \tikzstyle{bplus}=[rectangle split, rectangle split horizontal, rectangle split ignore empty parts, draw]
        \tikzstyle{every node}=[bplus]
        \tikzstyle{level 1}=[sibling distance=60mm]
        \tikzstyle{level 2}=[sibling distance=15mm]
        \node {9 \nodepart{two} 14} [->]
        child {node {7}
            child {node {6}}
            child {node {8}}
        }
        child {node {11}
            child {node {10}}
            child {node {12 \nodepart{two} 13}}
        }
        child {node {16 \nodepart{two} 19}
            child[sibling distance=25mm] {node {15 \nodepart{two}}}
            child {node {17 \nodepart{two} 18}}
            child[sibling distance=20mm] {node {20}}
        };
    \end{tikzpicture} \\ [5mm]
\end{center}

\paragraph{Zadanie 12} W B-drzewie o t = 10:
\begin{enumerate}[label=(\alph*)]
    \item ile kluczy może zawierać korzeń (podaj przedział), \\
    Korzeń zawiera od 1 do 19 kluczy. (max $2t-1$)
    \item ile dzieci może mieć korzeń (podaj przedział), \\
    Korzeń może mieć od 2 do 20 dzieci. (min $t$ max $2t$)
    \item ile kluczy może mieć potomek korzenia (podaj przedział), \\
    Potomek korzenia może mieć od 9 do 19 kluczy. (min $t-1$ max $2t-1$)
    \item ile dzieci może mieć potomek korzenia (podaj przedział), \\
    Potomek korzenia może mieć od 10 do 20 dzieci. (min $t$ max $2t$)
    \item ile maksymalnie węzłów może być na $k$-tym poziomie (przyjmując, że korzeń to poziom $0$) \\
    Na k-tym poziomie może być maksymalnie $(2t)^k$ węzłów.
    \item ile łącznie kluczy może być na $k$-tym poziomie (podaj przedział). \\
    Nie licząc korzenia dla którego minimum to 1 klicz to na k-tym poziomie może być od $2(t-1)t^{k-1}$ do $(2t-1)(2t)^{k}$ kluczy. (min $(2min)t^{k-1}$ max $(max)t^{k}$
\end{enumerate}

\end{document}