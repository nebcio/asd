\documentclass[18pt]{extarticle}

\usepackage{amsmath}
\usepackage{amssymb}
\usepackage{enumitem}
\usepackage[polish]{babel}
\usepackage[T1]{fontenc}
\usepackage[utf8]{inputenc}
\usepackage[margin=1.9cm]{geometry}
\usepackage[edges]{forest}
\usepackage{listings}
\usepackage{multicol}
\usepackage{tgbonum}
\usepackage{xcolor}

\begin{document}
\large
\fontfamily{cmss}\selectfont

\title{Algorytmy i struktury danych (Lista 7)}
\date{}
\maketitle

\text{ Jakie informacje przechowujemy w węźle B-drzewa? Podaj definicję B-drzewa.} \\

\begin{lstlisting}
    struct node {
        int key;
        node* left;
        node* right;
        node(int k, node* l, node* r):key(k),left(l),right(r){}
    };
\end{lstlisting}
\paragraph{Zadanie 1} Warunki klucza drzewa BST
\begin{enumerate}
    \item Lewa strona zawiera wartości mniejsze od rodzica
    \item Prawa większe lub równe
    \item Llucz z wartością
\end{enumerate}
\paragraph{Zadanie 3} 

\end{document}