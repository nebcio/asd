\documentclass[18pt]{extarticle}

\usepackage{amsmath}
\usepackage{amssymb}
\usepackage{enumitem}
\usepackage[polish]{babel}
\usepackage[T1]{fontenc}
\usepackage[utf8]{inputenc}
\usepackage[margin=1.9cm]{geometry}
\usepackage[edges]{forest}
\usepackage{listings}
\usepackage{multicol}
\usepackage{tgbonum}
\usepackage{xcolor}

\begin{document}
\large
\fontfamily{cmss}\selectfont

\title{Algorytmy i struktury danych (Lista 6)}
\date{}
\maketitle

% \paragraph{quick sort, selekcja, sortowanie bez porównań}
\paragraph{Zadanie 1} Jakie informacje przechowujemy w węźle drzewa czerwono-czarnego? Zadeklaruj strukturę RBTnode tak, by dziedziczyła z BSTnode. Podaj definicję drzewa czerwono czarnego. \\


Oprócz lewego/prawego dziecka kolor węzła (czerwony lub czarny). 
Logika/definicja: drzewo czerwono-czarne jest drzewem BST, w którym każdy węzeł ma dodatkowo kolor (czerwony lub czarny). Węzły czerwone mają 2 czarnych dzieci. Korzeń jest czarny. Każda ścieżka od korzenia do liścia ma tę samą liczbę czarnych węzłów. Każdy liść jest czarny (wartownik) a czerwony węzeł nie może mieć czerwonego dziecka. \\

\begin{lstlisting}
struct BSTnode:
    int key;	 
    node *left;	 
    node *right;
end

struct RBTnode : BSTnode:
    bool color;	 
end
\end{lstlisting}

\paragraph{Zadanie 2} (a) Jaka może być minimalna, a jaka maksymalna ilość kluczy w drzewie czerwono-czarnym o ustalonej czarnej wysokości równej hB?
(b) Znajdź maksymalną i minimalną wartość stosunku ilości węzłów czerwonych do czarnych w drzewie czerwono-czarnym.

\paragraph{Zadanie 3} Uzasadnij posługując się rysunkiem i opisem, że operacje wykonywane w trakcie wstawiania do drzewa czerwono-czarnego (rotacja i przekolorowanie) nie zmieniają ilości czarnych węzłów, na żadnej ścieżce od korzenia do liścia. \\


Przekolorowanie zwiększa liczbę czarnych węzłów dopiero gdy korzeń zastaje przekolorowany na czerwono (i w nastepnym kroku na czarno) \\
\begin{center}
    \begin{forest}
        [, phantom, for tree={circle, minimum size=3ex, inner sep=1pt, s sep=5mm, l sep=0mm, l=0mm, anchor=south, fill=black, text=white},
        [Z[U, fill=red[A, fill=red][, phantom]][Z, fill=red]]
        [Z, fill=red[U[A, fill=red][, phantom]][Z]]]
        [Z[U[A, fill=red][, phantom]][Z]]
    \end{forest}
\end{center}
W przypadku rotacji, nawet jeśli chwilowo zmieni się liczba czarnych węzłów na gałęzi, to po wykonaniu drugiej rotacji liczba czarnych węzłów wróci do poprzedniej wartości lub naprawimy ją przekolorowaniem. \\
Samo przekolorowywanie natomiast polega \\
\begin{center}
    \begin{forest}
        [, phantom, for tree={circle, minimum size=3ex, inner sep=1pt, s sep=5mm, l sep=0mm, l=0mm, anchor=south, fill=black, text=white},
        [Z[U[A, fill=red[, phantom][N, fill=red]][, phantom]][Z]]
        [Z[U[N, fill=red[A, fill=red][, phantom]][, phantom]][Z]]
        [Z[N[A, fill=red][U]][Z]]
        [Z[N[A, fill=red][U, fill=red]][Z]]
        [Z[N[A, fill=red][U, fill=red[N, fill=red][, phantom]]][Z]]
        [Z[N, fill=red[A][U[N, fill=red][, phantom]]][Z]]
        [Z[N, fill=red[A[, phantom][A, fill=red]][U[N, fill=red][, phantom]]][Z]]
        ]
    \end{forest}
\end{center}


\paragraph{Zadanie 4} (a) Narysuj poprawne drzewo czerwono czarne w którym na lewo od korzenia jest 1 węzeł a na prawo 7 węzłów.
(b) Czy istnieje poprawne drzewo czerwono czarne, w którym na lewo od korzenia będzie 100 razy mniej węzłów niż na prawo od korzenia?

\pagebreak
\paragraph{Zadanie 5} W poniższym drzewie czerwono-czarnym (czarne węzły oznaczono nawiasem kwadratowym), wstaw do niego 10 i usun z wyjsciowego drzewa 1:

\begin{multicols}{2}
\begin{lstlisting}
        [5] 
   (3)     [11] 
 [1] [4]  (9) 


         [5] 
     (3)     [11] 
 [1]   [4]  (9) 
             (10) 
 
         [5] 
     (3)      [11] 
 [1]   [4]   (10) 
            (9) 
 
         [5] 
     (3)      [10] 
 [1]   [4]   (9) (11)
 
         [5] 
     (3)      (10) 
 [1]   [4]   [9] [11]
             
\end{lstlisting}

\columnbreak
\begin{lstlisting}
         [5] 
     (3)      (10) 
      [4]   [9] [11]

         [5] 
     [3]      (10) 
      (4)   [9] [11]
\end{lstlisting}
\end{multicols}


\paragraph{Zadanie 6} (3 pkt.) Do pustego drzewa czerwono-czarnego wstaw kolejno 20 przypadkowych kluczy. Następnie usuń je w tej samej kolejności w jakiej wstawiałeś. Przypadkowymi kluczami są kolejne litery Twojego nazwiska, imienia i adresu. Zadanie wykonujemy na kartce (lub w pliku) i oddajemy prowadzącemu. Zadanie jest obowiązkowe

\paragraph{Zadanie 7} Analizując kod programu RBT.h udowodnij, że w trakcie wstawiania do drzewa czerwono-czarnego wykonają się co najwyżej dwie rotacje. Czy tak samo jest w przypadku usuwania


\paragraph{Zadanie 8} Uzasadnij, że rozmiar stosu (n = 100) przyjęty w procedurach insert i remove w pliku RBnpnr.h nigdy nie okaże się za mały


\end{document}