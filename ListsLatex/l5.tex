\documentclass[18pt]{extarticle}

\usepackage{amsmath}
\usepackage{amssymb}
\usepackage{enumitem}
\usepackage[polish]{babel}
\usepackage[T1]{fontenc}
\usepackage[utf8]{inputenc}
\usepackage[margin=1.9cm]{geometry}
\usepackage[edges]{forest}
\usepackage{listings}
\usepackage{multicol}
\usepackage{tgbonum}
\usepackage{xcolor}

\begin{document}
\large
\fontfamily{cmss}\selectfont

\title{Algorytmy i struktury danych (Lista 5)}
\date{}
\maketitle

% \paragraph{quick sort, selekcja, sortowanie bez porównań}
\paragraph{Zadanie 1} Udowodnij, że jeśli dla pewnego ustalonego q, takiego że 1/2 < q < 1, podczas sortowania szybkiego, procedura partition, na każdym poziomie rekurencji podzieli elementy tablicy w stosunku q : (1 - q) to algorytm wykona się w czasie O(n log n). Wskazówka: udowodnij, że głębokość rekurencji nie przekroczy - log n/ log q i zaniedbaj błędy zaokrągleń do wartości całkowitych.

\begin{multicols*}{3}
\begin{forest}
    for tree={math content, minimum size=3ex, inner sep=1pt, s sep=7mm, anchor=south}
    [n[(1-q)n][qn[(1-q)qn][q^2n[(1-q)q^kn, edge=dashed][q^kn, edge=dashed]]]]
\end{forest}

\columnbreak
\begin{gather*}
    q^kn = 1 \\
    q^k = \frac{1}{n} \\
    \log_q q^k = \log_q n^{-1}
\end{gather*}

\columnbreak
\begin{gather*}
    k = -\log_q n = \\
    = - \frac{\log_2 n}{\log_2 q} \\
    \log_2 \frac{1}{2} < \log_2 q < \log_2 1 \\
    -1 < \log_2 q < 0 \\
    k = -\frac{\log n}{\log q}
\end{gather*}
\end{multicols*}

\paragraph{Zadanie 2} Ile porównań (zapisz wyniki w notacji O) wykona algorytm quicksort z procedurą partition w wersji Hoare'a, a ile w wersji z procedurę partition w wersji Lomuto dla danych: (a) posortowanych rosnąco, (b) posortowanych malejąco, (c) o identycznych kluczach?
\begin{multicols*}{3}
Posortowane rosnąco: \\
H = O(n log n), L = O($n^2$). \\

\columnbreak
Dla posortowanych malejąco: \\
H = O(n log n), L = O($n^2$). \\

\columnbreak
Dla identycznych kluczy: \\
H = O($n^2$), L = O($n^2$) \\ 
\end{multicols*}

\paragraph{Zadanie 3} Napisz wzór na numer kubełka, do którego należy wrzucić liczbę x w sortowaniu kubełkowym, jeśli kubełków jest n, a elementy tablicy mieszczą się przedziale (a, b). Numeracja zaczyna się od 0. \\


\begin{align*}
    k = \left \lfloor \frac{n}{b - a} \cdot (x - a) \right \rfloor
\end{align*}

\paragraph{Zadanie 4} Dla jakich danych sortowanie metodą kubełkową ma złożoność O(n2)? \\


Dane, które są rozłożone mocno nierównomiernie np. grupują się w jeden kubełek.

\paragraph{Zadanie 5} Jak obliczyć k-tą od końca cyfrę w liczbie x? Jak obliczyć ilość cyfr liczby x? Przyjmujemy układ dziesiętny. Jak wyniki zmienią się w układzie pozycyjnym gdzie różnych cyfr jest m a ich wartości x należą do przedziału 0 <= x < m?


\begin{gather*}
    k_w = \left \lfloor \frac{x}{10^k} \right \rfloor \mod 10 \\ \\
    \text{dla }x = 123456, k = 2, k_w = 4 \\ 
    \frac{123456}{10^2} = 1234.56 \rightarrow 1234 \mod 10 = 4 \\ \\
    \text{m różnych cyfr:} \\
    k_w = \left \lfloor \frac{x}{m^k} \right \rfloor \mod m \\ \\
    \text{dla }x = 123456, k = 3, m = 6, k_w = 3 \\
    \frac{123456}{6^3} = 123.456 \rightarrow 123 \mod 6 = 3
\end{gather*}

\pagebreak
\paragraph{Zadanie 6} Posortuj metodą sortowania pozycyjnego liczby: 101, 345, 103, 333, 432, 132, 543, 651, 791, 532, 987, 910, 643, 641, 12, 342, 498, 987, 965, 322, 121, 431, 350. W pisemnym rozwiązaniu pokaż, jak wygląda zawartość kolejek, za każdym razem, gdy tablica wyjściowa jest pusta i wszystkie liczby znajdują się w kolejkach, oraz jak wygląda tablica wyjściowa, za każdym razem, gdy sortowanie ze względu na kolejną cyfrę jest już zakończone.
\begin{align*}
    0: & \left(910, 350\right) \\
    1: & \left(101, 651, 791, 641, 121, 431\right) \\
    2: & \left(432, 132, 532, 12, 342, 322\right) \\
    3: & \left(103, 333, 543, 643\right) \\
    4: & \left(\right) \\
    5: & \left(345, 965\right) \\
    6: & \left(\right) \\
    7: & \left(987, 987\right) \\
    8: & \left(498\right) \\
    9: & \left(\right)
\end{align*}
(910, 350, 101, 651, 791, 641, 121, 431, 432, 132, 532, 12, 342, 322, 103, 333, 543, 643, 345, 965, 987, 987, 498)

\begin{align*}
    0: & \left(101, 103\right) \\
    1: & \left(910\right) \\
    2: & \left(121, 12, 322\right) \\
    3: & \left(431, 432, 132, 532, 333\right) \\
    4: & \left(641, 342, 543, 643, 345\right) \\
    5: & \left(350, 651\right) \\
    6: & \left(965\right) \\
    7: & \left(\right) \\
    8: & \left(987, 987\right) \\
    9: & \left(791, 498\right)
\end{align*}
(101, 103, 910, 121, 12, 322, 431, 432, 132, 532, 333, 641, 342, 543, 643, 345, 350, 651, 965, 987, 987, 791, 498)

\begin{align*}
    0: & \left(12\right) \\
    1: & \left(101, 103, 121, 132\right) \\
    2: & \left(\right) \\
    3: & \left(322, 333, 342, 345, 350\right) \\
    4: & \left(431, 432, 498\right) \\
    5: & \left(532, 543\right) \\
    6: & \left(641, 643, 651\right) \\
    7: & \left(791\right) \\
    8: & \left(\right) \\
    9: & \left(910, 965, 987, 987\right)
\end{align*}
(12, 101, 103, 121, 132, 322, 333, 342, 345, 350, 431, 432, 498, 532, 543, 641, 643, 651, 791, 910, 965, 987, 987)

\paragraph{Zadanie 7} (2pkt) Które z procedur sortujących:
\begin{enumerate}
    \item insertionSort - stabilny, zachowuje kolejność bo przestawia elementy od lewej do prawej strony nie zmieniającą tych o tej samej wartości
    \item quickSort - niestabilny, [2, 1, 1] 1 z ostatniego indeksu zostanie zamieniona z 2 z 0 indeksu, potem gdy weźmiemy [1,2] to nie będzie już swapa 
    \item heapSort - niestabilny, [1,1] budowanie kopca nic nie zmieniło, ale na etapie końcowym zamieni nam 1 miejscami i wywoła przesiewanie, które isę przerwie i nic nie zmieni ;')
    \item mergeSort - stabilny, ponieważ przy łączeniu dwóch posortowanych tablic zachowuje kolejność elementów o tej samej wartości
    \item countingSort - stabilny, ponieważ na końcowym etapie idąc od końca tablicy wejściowej wpisujemy elementy na odpowiednie ideksy przez co jeśli dana liczba była później od innej to tak pozostanie (gdy mają te same wartości)
    \item radixSort - stabilny, ponieważ wykorzystuje się kolejki FIFO (first in first out)
    \item bucketSort - stabilny, idziemy od końca i do bufora wpisujemy od ostatniegi indeksu gdzie elementy z danego kubełku powinny być w tablicy wyjściowej
\end{enumerate}
są stabilne? W każdym przypadku uzasadnij stabilność lub znajdź konkretny przykład danych, dla których algorytm nie zachowa się stabilnie.

\paragraph{Zadanie 8} Napisz funkcję void \verb+counting_sort(node* lista, int m);+ sortującą przez zliczanie listę linkowaną liczb całkowitych nieujemnych mniejszych od $m$. Procedura nie powinna usuwać ani tworzyć nowych węzłów, tylko sprytnie zmieniać pola next wykorzystując tylko $O(m)$ dodatkowej pamięci na wskaźniki.

\paragraph{Zadanie 9} (algorytm Hoare'a) Korzystając funkcji \verb+int partition(int t[], int n)+ znanej z algorytmu sortowania szybkiego napisz funkcję \verb+int kty(int t[], int n)+, której wynikiem będzie $k$-ty co do wielkości element początkowo nieposortowanej tablicy \verb+t+. Średnia złożoność Twojego algorytmu powinna wynieść $O(n)$.


\end{document}