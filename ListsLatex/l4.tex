\documentclass[18pt]{extarticle}

\usepackage{amsmath}
\usepackage{amssymb}
\usepackage{enumitem}
\usepackage[polish]{babel}
\usepackage[T1]{fontenc}
\usepackage[utf8]{inputenc}
\usepackage[margin=1in]{geometry}
\usepackage[edges]{forest}
\usepackage{listings}
\usepackage{multicol}
\usepackage{tgbonum}
\usepackage{xcolor}

\begin{document}
\large
\fontfamily{cmss}\selectfont

\title{Algorytmy i struktury danych (Lista 4)}
\date{}
\maketitle

\paragraph{Zadanie 7} Metodą jak na wykładzie, udowodnij, że procedura \texttt{buildHeap} działa w czasie O(n).


\paragraph{Zadanie 8} Udowodnij, że wysokość kopca n-elementowego wynosi $ \lfloor log2 n\rfloor + 1 $
\begin{center}
    \begin{forest}
        for tree={circle, draw, minimum size=3ex, inner sep=1pt, s sep=7mm, anchor=south, fill=green!50}
        [7[6[4][3]][5[2][1]]]
    \end{forest} \\
    i = ostatnie dziecko z dziećmi = $n/2-1$ \\
    $i = 0$\\
    $2(2(2 * 0+1)+1)+1... < n$
\end{center}
\begin{gather*}
    \text{Maksymalna ilość węzłów w kopcu o wysokości $h$:} \\
    n(h) = 2^h - 1 \\
    n(h-1) = 2^{h - 1} - 1 \\
    \text{Minimalna ilość węzłów w kopcu o wysokości $h$:} \\
    n(h-1) + 1 = 2^{h - 1} - 1 + 1 = 2^{h - 1} \\
    \text{Ilość węzłów w kopcu o wysokości $h$:} \\
    2^{h - 1} \leq n < 2^h \\
    h-1 \leq \log_2 n < h \\
    h \leq \log_2 n + 1 < h + 1 \\
    \lfloor\log_2 n\rfloor + 1
\end{gather*}

\paragraph{Zadanie 9} (2 pkt) Podaj ideę algorytmu, jak przy pomocy struktury kopca, złączyć k posortowanych list jednokierunkowych o łącznej ilości elementów n, w jedną posortowaną listę, za używając nie więcej niż $3n$ $log_2 k$ porównań \\


\text{k list po } $\frac{n}{k}$
\text{ elementów}
\begin{enumerate}
    \item \text{kopiec z pierwszych elementów list}
    \item korzeń wraca do listy i jest zastępowany kolejnym elementem z rodzimej listy dla korzenia
    \item przesiewamy -> $2log_2k$ porównań dla jednego elementu
    \item n elementów więc w sumie $2n$ $log_2k$
    \item wyciągamy elementy z kopca
\end{enumerate}


\paragraph{Zadanie 10} W pliku spis.txt umieszczone są w przypadkowej kolejności wszystkie liczby całkowite od 1 do n za wyjątkiem jednej (n jest bardzo duże). Jak wyliczyć brakującą liczbę w czasie liniowym nie wykorzystując dodatkowej pamięci plikowej ani RAM za wyjątkiem kilku zmiennych typu int? Wskazówka: w C++ działania + - * na liczbach całkowitych obywają się modulo $2^{32}$

\paragraph{Zadanie 11} Niech Fn oznacza ilość różnych kształtów drzew binarnych o n węzłach. Rysując drzewa, łatwo sprawdzić, że \texttt{F0 = 1, F1 = 1, F2 = 2, F3 = 5, itd.} Nie korzystając z internetu:


\begin{enumerate}[label=(\alph*)]
    \item Znajdź wzór wyrażający $F_n$ przez $F_0, F_1, F_2 \dots, F_{n-1}$ dla $n = 2, 3, 4$ a potem ogólnie. \\
        \text{$F_0 = 1, F_1 = 1$} \\
        \text{$F_2 = 2 = F_0F_1 + F_1F_0$} \\
        \text{$F_3 = 5 = F_0F_2 + F_1F_1 + F_2F_0$} \\
        \text{$F_4 = 14 = F_0F_3 + F_1F_2 + F_2F_1 + F_3F_0$} \\
        \text{$F_n = \sum_{i=1}^{n} F_{i-1}F_{n-i}$} \\
        \text{Wymnażamy prawą stronę danego węzła przez lewą stronę a potem na odwrót co po zsumowaniu daje nam liczbę możliwości dla danego węzła. i - 1 bo 1 to korzeń.}
    \item Zaprojektuj (na kartce) procedurę, która oblicza kolejne wyrazy ciągu $F_n$, zapisuje
          je w tablicy i korzysta z nich przy obliczaniu następnych wyrazów.
          \begin{lstlisting}
                F[0] = 1
                F[1] = 1
                for i = 2 to n:
                    F[i] = 0
                    for j = 1 to i:
                        F[i] += F[j-1] * F[i-j]
          \end{lstlisting}
    \item Przeanalizuj ile mnożeń trzeba wykonać, by obliczyć wyrazy od $F_1$ do $F_n$. Czy da
          się ją zapisać w postaci $O(n^k)$ dla pewnego $k$?
          \begin{gather*}
                \text{Ilośc mnożeń: } \sum_{i=1}^{n} i = \frac{n(1+n)}{2} = O(n^2)
          \end{gather*}
    \item Jaka byłaby złożoność algorytmu rekurencyjnego, który nie korzysta z wartości
          zapisanych w tablicy, tylko oblicza je ponownie. Czy da się ją zapisać jako $O(n^k)$?
          \begin{gather*}
                \sum_{k=1}^{n} = \frac{k(1+k)}{2} = \frac{n(n+1)(n+2)}{6} = O(n^3)
          \end{gather*}
\end{enumerate}

\paragraph{Zadanie 12} Wykonaj zadanie 11 (a) (b) (c) dla drzew trynarnych (dzieci: left, down, right).
\end{document}