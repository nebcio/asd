\documentclass[18pt]{extarticle}

\usepackage{amsmath}
\usepackage{amssymb}
\usepackage{enumitem}
\usepackage[polish]{babel}
\usepackage[T1]{fontenc}
\usepackage[utf8]{inputenc}
\usepackage[margin=1in]{geometry}
\usepackage[edges]{forest}
\usepackage{listings}
\usepackage{multicol}
\usepackage{tgbonum}
\usepackage{xcolor}

\begin{document}
\large
\fontfamily{cmss}\selectfont

\title{Algorytmy i struktury danych (Lista 4)}
\date{}
\maketitle

\paragraph{Zadanie 6}
\text{Rysunki przedstawiają kopce tuż po wywołaniu funkcji \texttt{przesiej}}
\begin{multicols*}{2}
    \begin{center}
        wejściowa:\\
        {5 3 17 10 \textcolor{red}{84} 19 6 22 9 14 3}\\[12ex]
        wyjściowa:\\
        {5 3 17 10 84 19 6 22 9 14 3}\\[1ex]

        wejściowa:\\
        {5 3 17 \textcolor{red}{10} 84 19 6 22 9 14 3}\\[12ex]
        wyjściowa:\\
        {5 3 17 22 84 19 6 10 9 14 3}\\[1ex]

        wejściowa:\\
        {5 3 \textcolor{red}{17} 10 84 19 6 22 9 14 3}\\[12ex]
        wyjściowa:\\
        {5 3 19 22 84 17 6 10 9 14 3}\\[1ex]

        wejściowa:\\
        {5 \textcolor{red}{3} 19 10 84 17 6 22 9 14 3}\\[12ex]
        wyjściowa:\\
        {5 84 19 22 3 17 6 10 9 14 3}\\[1ex]

    \columnbreak
        \begin{forest}
            for tree={circle, draw, minimum size=3ex, inner sep=1pt, s sep=7mm, anchor=south, fill=blue!30}
            [5[3[10[22][9]][84,fill=red!50[14][3]]][17[19][6]]]
        \end{forest}

        \begin{forest}
            for tree={circle, draw, minimum size=3ex, inner sep=1pt, s sep=7mm, anchor=south, fill=blue!30}
            [5[3[10,fill=red!50[22][9]][84[14][3]]][17[19][6]]]
        \end{forest}

        \begin{forest}
            for tree={circle, draw, minimum size=3ex, inner sep=1pt, s sep=7mm, anchor=south, fill=blue!30}
            [5[3[22[10][9]][84[14][3]]][17,fill=red!50[19][6]]]
        \end{forest}

        \begin{forest}
            for tree={circle, draw, minimum size=3ex, inner sep=1pt, s sep=7mm, anchor=south, fill=blue!30}
            [5[3,fill=red!50[22[10][9]][84[14][3]]][19[17][6]]]
        \end{forest}
    \end{center}
\end{multicols*}
\pagebreak
\begin{multicols*}{2}
    \begin{center}
        wejściowa:\\
        {5 84 19 22 \textcolor{red}{3} 17 6 10 9 14 3}\\[12ex]
        wyjściowa:\\
        {5 84 19 22 14 17 6 10 9 3 3}\\[1ex]

        wejściowa:\\
        {\textcolor{red}{5} 84 19 22 14 17 6 10 9 3 3}\\[11ex]
        wyjściowa:\\
        {84 5 19 22 14 17 6 10 9 3 3}\\[1ex]

        wejściowa:\\
        {84 \textcolor{red}{5} 19 22 14 17 6 10 9 3 3}\\[11ex]
        wyjściowa:\\
        {84 22 19 5 14 17 6 10 9 3 3}\\[1ex]

        wejściowa:\\
        {84 \textcolor{red}{5} 19 22 14 17 6 10 9 3 3}\\[11ex]
        wyjściowa:\\
        {84 22 19 10 14 17 6 5 9 3 3}\\[1ex]

        ostateczny kopiec po skończeniu powyższego wywołania \texttt{przesiej}
    
        \columnbreak
        \begin{forest}
            for tree={circle, draw, minimum size=3ex, inner sep=1pt, s sep=7mm, anchor=south, fill=blue!30}
            [5[84[22[10][9]][3,fill=red!50[14][3]]][19[17][6]]]
        \end{forest}

        \begin{forest}
            for tree={circle, draw, minimum size=3ex, inner sep=1pt, s sep=7mm, anchor=south, fill=blue!30}
            [5,fill=red!50[84[22[10][9]][14[3][3]]][19[17][6]]]
        \end{forest}

        \begin{forest}
            for tree={circle, draw, minimum size=3ex, inner sep=1pt, s sep=7mm, anchor=south, fill=blue!30}
            [84[5,fill=red!50[22[10][9]][14[3][3]]][19[17][6]]]
        \end{forest}

        \begin{forest}
            for tree={circle, draw, minimum size=3ex, inner sep=1pt, s sep=7mm, anchor=south, fill=blue!30}
            [84[22[5,fill=red!50[10][9]][14[3][3]]][19[17][6]]]
        \end{forest}

        \begin{forest}
            for tree={circle, draw, minimum size=3ex, inner sep=1pt, s sep=7mm, anchor=south, fill=blue!30}
            [84[22[10[5][9]][14[3][3]]][19[17][6]]]
        \end{forest}

    \end{center}
\end{multicols*}
\end{document}